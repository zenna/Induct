% TLP2esam.tex / sample pages for TLP
% v2.11, released 6-nov-2002

\documentclass{tlp}
\usepackage{aopmath}

\begin{document}
\bibliographystyle{acmtrans}

\title{The RL Super-Learning Bot-Machine-Program}

\author[Z. Tavares, M. Siegel]
{TAVARES ZENNA \\
MIT\\
E-mail: zenna@mit.edu
}

% \author[M. SIEGEL]
% {SIEGEL MAX \\
% MIT\\
% E-mail: maxs@mit.edu
% }

\pagerange{\pageref{firstpage}--\pageref{lastpage}}

\maketitle

\label{firstpage}

\begin{abstract}
%
Boom!

\end{abstract}

The idea is to simultaneously learn a model and use that model to plan in some set of domains.
The model is represented as a probabilistic program P we can sample new world states from.
That is, a model is a random variable from a current state and some random input to a representation of the world and the reward.  If there are $S$ states, the set of all of states = \mathcal{S} = \mathcal{P}(S).  A model is then:

\[
P : \Omega \times \mathcal{S} \to \mathcal{S} \times R
\]

Suppose the set of all type-consistent well forms programs defines a language $L$.
We have an initial set of actions $A_init$ which are syntactic transformations on $P$. I.e.,

\[
a: \mathcal{L} \to \mathcal{L}.
\]

\subsection{Building a model}
We build the model by applying syntax transformations to the model
A model is evaluated by how well it performs in the 


\subsection{Domain}
We wil use the Arcade Learning Environment

\bibliography{tlp2esam}

\end{document}



